%%%%%%%%%%%%%%%%%%%%%%%%%%%%%%%%%%%%%%%%%%%%%%%%%%%%%%%%%%%%
%%%%%%%%%%%%%%%%%%%%%%%%%%%%%%%%%%%%%%%%%%%%%%%%%%%%%%%%%%%%
%%%%%%%%%%%%%%%%%%%%%%%%%%%%%%%%%%%%%%%%%%%%%%%%%%%%%%%%%%%%
%%%%%%%%%%%%%%%%%%%%%%%%%%%%%%%%%%%%%%%%%%%%%%%%%%%%%%%%%%%%
%%%%%%%%%%%%%%%%%%%%%%%%%%%%%%%%%%%%%%%%%%%%%%%%%%%%%%%%%%%%
\documentclass[12pt]{article}
\usepackage{epsfig}
\usepackage{times}
\usepackage{fancyhdr}
\usepackage{pslatex}
\usepackage{amsmath}
\usepackage{mathrsfs}
\usepackage[dvipsnames]{xcolor}
\usepackage[hidelinks]{hyperref}%renewcommand{\topfraction}{1.0}
\renewcommand{\topfraction}{1.0}
\renewcommand{\bottomfraction}{1.0}
\renewcommand{\textfraction}{0.0}
\setlength {\textwidth}{6.6in}
\hoffset=-1.0in
\oddsidemargin=1.00in
\marginparsep=0.0in
\marginparwidth=0.0in                                                                               
\setlength {\textheight}{9.0in}
\voffset=-1.00in
\topmargin=1.0in
\headheight=0.0in
\headsep=0.00in
\footskip=0.50in                                         
\setcounter{page}{1}
\begin{document}
\def\pos{\medskip\quad}
\def\subpos{\smallskip \qquad}
\newfont{\nice}{cmr12 scaled 1250}
\newfont{\name}{cmr12 scaled 1080}
\newfont{\swell}{cmbx12 scaled 800}
%%%%%%%%%%%%%%%%%%%%%%%%%%%%%%%%%%%%%%%%%%%%%%%%%%%%%%%%%%%%
%     DO NOT CHANGE ANYTHING ABOVE THIS LINE
%%%%%%%%%%%%%%%%%%%%%%%%%%%%%%%%%%%%%%%%%%%%%%%%%%%%%%%%%%%%
%     DO NOT CHANGE ANYTHING ABOVE THIS LINE
%%%%%%%%%%%%%%%%%%%%%%%%%%%%%%%%%%%%%%%%%%%%%%%%%%%%%%%%%%%%
%     DO NOT CHANGE ANYTHING ABOVE THIS LINE
%%%%%%%%%%%%%%%%%%%%%%%%%%%%%%%%%%%%%%%%%%%%%%%%%%%%%%%%%%%%

\begin{center}
{\Large
{PHYSICS  20323/60323: Fall 2024 - LaTeX Example}
}\\
%%%%%%%%%%%%%%%%%%%%%%%%%%%%%%%%%%%%%%%%%%%%%%%%%%%%%%%%%%%%
%%%%%%%%%%%%%%%%%%%%%%%%%%%%%%%%%%%%%%%%%%%%%%%%%%%%%%%%%%%%
\end{center}
%%%%%%%%%%%%%%%%%%%%%%%%%%%%%%%%%%%%%%%%%%%%%%%%%%%%%%%%%%%%
% Section Heading
%%%%%%%%%%%%%%%%%%%%%%%%%%%%%%%%%%%%%%%%%%%%%%%%%%%%%%%%%%%%
\noindent 1. An electron is found to be in a spin state (in the \textit{z}-basis): $\chi=A \begin{pmatrix}3i \\ 4 \end{pmatrix}$\\

(a) (5 points) Determine the possible values of \textit{A} such that the state is normalized. \\

\indent (b) (5 points) Find the expectation values of the operators {\color{red}\textit{$S_x$}}, {\color{purple}\textit{$S_y$}}, {\color{orange}\textit{$S_z$}}, and \textit{${\Vec{S}}^2$}.\\

\indent{The matrix representations in the \textit{z}-basis for the components of electron spin operators are given \indent by:} \\
\\
\indent {\color{red}{{$\mathbf{S_x}$} = {$\tfrac{\hbar}{2}$}\begin{pmatrix}0 \ 1 \\ 1 \ 0 \end{pmatrix}};} \qquad {\color{purple}{{$\mathbf{S_y}$} = {$\tfrac{\hbar}{2}$}\begin{pmatrix}0 \ \textit{-i} \\ \textit{i} \ 0 \end{pmatrix}};} \qquad {\color{orange}{{$\mathbf{S_z}$} = {$\tfrac{\hbar}{2}$}\begin{pmatrix} $1$ \ \  $0$ \\ $0$ \ $-1$ \end{pmatrix}};} \\


2. The average electrostatic field in the earth's atmosphere in fair weather is approximately given: \\
\begin{equation}
    {\Vec{E}} = E_0(Ae^{-{\alpha}z}+Be^{-{\beta}z}){\hat{z}},
\end{equation}
\indent where $A$, $B$, {$\alpha$}, {$\beta$} are positive constants and $z$ is the height above the (locally flat) earth surface. \\
\\
\indent (a) (5 points) Find the average charge density in the atmosphere as a function of height \\

\indent (b) (5 points) Find the electric potential as a function of height above the earth.\\
\\

\noindent 3. \textbf{The following questions refer to the stars in the Table below.} \\
\indent Note: There may be multiple answers. 
\begin{table}[h]
    \begin{tabular}{|c|c|c|c|c|c|}
        \hline
        Name & Mass & Luminosity & Lifetime & Temperature & Radius \\ \hline
        $\beta$ Cyg. & $1.3 \, M_{\odot}$ & $3.5 \, L_{\odot}$ &  &  & $1 \, R_{\odot}$ \\ \hline
        $\alpha$ Cen. & $1.0 \, M_{\odot}$ &  &  &  & \\ \hline
        $\eta$ Car. & $60. \, M_{\odot}$ & $10^{6} \, L_{\odot}$ & $8.0 \times 10^{5}$ years &  & \\ \hline
        $\epsilon$ Eri. & $6.0 \, M_{\odot}$ & $10^{3} \, L_{\odot}$ &  & $20,000 \, \text{K}$ & \\ \hline
        $\delta$ Scu. & $2.0 \, M_{\odot}$ &  & $5.0 \times 10^{8}$ years &  & $2 \, R_{\odot}$ \\ \hline
        $\gamma$ Del. & $0.7 \, M_{\odot}$ &  & $4.5 \times 10^{10}$ years & $5000 \, \text{K}$ & \\ \hline
    \end{tabular}
\end{table}

\indent (a) (4 points) Which of these stars will produce a planetary nebula. \\

\indent (b) (4 points) Elements heavier than \textit{Carbon} will be produced in which stars.\\

\end{document}
